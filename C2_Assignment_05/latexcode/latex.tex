\documentclass[twocolumn]{article}
\usepackage[utf8]{inputenc}
\usepackage{amsmath}
\usepackage{graphicx}
\font\myfont=cmr12 at 30pt
\title{\vspace{-2cm} \textbf{{\myfont Design and Analysis of Algorithms}} \\
\\Group-2
}


\author{Nidhi kamewar    \: \: \:  \:     \:                         Rishi Mukesh Gupta          \: \: \:  \:     \:             Pechetti Venkata Karthik\\    

    IIT2019189     \: \: \:  \:     \:   \: \: \:  \:     \:                      IIT2019190     \: \: \:  \:     \: \:  \:    \: \:    \:                  IIT2019191    \: \: \: \:  \:     \:     


}



\begin{document}

\maketitle

\section{INTRODUCTION}
In this problem,we have to find all the substrings of a given string which are palindromic \\

As the given problem have Overlapping subproblems of recalculating the palindromic substrings of substrings in the given string and Optimal substructure property i.e the given problem can be solved by the solving the subproblems of the problem\\

Hence We can solve the given problem using Dynamic programming.\\

This report further contains -\\
\\
II. Algorithm Design \\
III. Algorithm Analysis\\
IV. Experimental Study\\
V. Conclusion\\

\section{ ALGORITHM DESIGN
}
\noindent
The algorithm we designed is based on dynamic programming Memorization(Top down) at first we check if the answer to the subproblem is already computed or not in the lookup table if present we return the value or else  we store the result of the problem(checking whether a substring of the string is true or false) in the (lookup table) 2D array.\\



\noindent
1.First,string S is passed as an input to the function countSubstrings\\
\noindent\\
2.The function initialises all the values of the lookup table to -1 which represents that result of the subproblem is not yet computed\\
\noindent\\
3.We initialise the value of count to 0 which indicates the number of palindromes of the string which will be incremented when a palindromic substring is found.\\
\noindent\\
4.The function iteratively checks the substrings by considering the strings from i to j where i will range from 0 to n-1 and  j will range from i+1 to n-1 
respectively  by calling the function isPal \\
\noindent\\
5.which checks if the substring of S between i to j is a substring or not by checking the lookup table and 
returning the respective value or computing the result and storing it in the lookup table.\\
\noindent\\
6.If the starting and the ending character of the string is the same then it will recursively call itself to check whether the substring from i+1 to j-1 is palindrome or not i.e, the substring in the middle.\\
\noindent\\
7.If The value of the starting index is greater then the ending index then it represents that the starting and ending characters of the substrings are matching at each recursive call of the funcion .\\
\noindent\\
8.The respective result will be stored in the lookup table.\\
\noindent\\
9.If the substring is a palindrome then the value of count is incremented by 1 it repeats till all the substring are checked and the value of the count is returned\\
\noindent\\
\\
\\
For example :\\
\noindent\\
Consider the string aabcca as S when passed to the function 
On each iteration the  strings\\
\\
i=0;\\
“aa”,“aab”,“aabc”,“aabcc”, “aabcca” ;\\
\\
i=1;\\
“ab”,“abc”,abcc”,“abcca”;\\
\\
i=2\\
“bc”,“bcc”,“bcca”;\\
\\
i=3\\
“cc”,”cca”;\\
\\
i=4\\
“ca”;\\
\noindent\\
are checked if they are substrings \\

Lookup table:





\begin{center}
	\begin{tabular}{ |c | c | c | c | c | c | c| }
	\hline
	 &0&1&2&3&4&5\\
	 \hline
	0&1&1&0&0&0&0\\
	\hline
1&-1&1&0&0&0&0\\
\hline
2&-1&-1&1&0&0&0\\
\hline
3&-1&-1&-1&1&1&0\\
\hline
4&-1&-1&-1&-1&1&0\\
\hline
5&-1&-1&-1&-1&-1&1\\
	\hline
	\end{tabular}
\end{center}



\noindent\\
Here in the above example the strings like abcc are not recomputed as the result of the subproblem is already computed when calculating the result of aabcca.\\
When the algorithm checks strings aa,cc the value of count is incremented by 1 and finally reaches 8 which is the total number of palindromic substrings of the given string “aabcca”.\\
i.e,a,a,b,c,c,c,a,aa,cc\\
\section{   PSEUDO CODE
}

int dp[1001][1001];\\
\textbf{Function} isPal(string s, int i, int j):\\
\\
          \textbf{\qquad if} \:i \textgreater \:j 	\textbf{then} \\
         \textbf{return }  1;\\
\textbf{end if }	\\
	\textbf{if } dp[i][j] != -1 \textbf{then} \\
	  \textbf{return } dp[i][j];\\
	\textbf{end if }\\
	\textbf{if } s[i] != s[j]  \textbf{then}\\
	    \textbf{return }  dp[i][j] = 0;\\
\textbf{end if }\\
	  \textbf{return }  dp[i][j] = isPal(s, i + 1, j - 1);\\
\noindent 
\\
\textbf{Function} countSubstrings(string s):\\
\\
\noindent 
	memset(dp, -1, sizeof(dp));\\
	int n ← s.length();\\
	int count ← 0;\\
	\textbf{for}  i ←  0 to n-1 \textbf{do}\\
	    	\textbf{for}  j ←  i  to n-1  \textbf{do}\\
		\textbf{if } isPal(s, i, j) \textbf{then} 	 \\
count++;\\
	  \textbf{return }  count;\\
\\
\textbf{Function} main():    \\
\\
    Get s;\\
    Print(countSubstrings(s))\\
\\
\\
\\
\section{ ALGORITHM ANALYSIS
}
Time complexity

The algorithm will have a O(n^2) time complexity where n is the length of the string. This is because all possible substrings are being visited and are checked whether they are a palindrome or not (in all the best,avg,worst cases). The total number of substrings is n*(n+1)/2 which is in the order of n^2. Using the memorization table, it can be checked whether a substring is palindrome or not in O(1). Hence, the total time complexity is O(n^2). 

\includegraphics[width=.45\textwidth]{2.jpeg}
The time complexity curve when N (string length against Time taken ) should be similar to the graph(Fig.1)(parabola)  between the Size of the given string(N) and time(y-axis)
Which represents  the time complexity of the Algorithm to be O(n^2)

Space complexity

The algorithm will have a O(n^2) space complexity where n is the length of the string.  This is because a memorization table (nxn 2d array) is used to store the results of the subproblems. 

Therefore,  O(n^2) auxiliary space is required.

\includegraphics[width=.45\textwidth]{1.jpeg}

The space complexity curve when N(string length against Space(lookup table size)) should be similar to the graph(Fig.2)(parabola)
Which represents  the Space complexity
 of the Algorithm to be O(n^2) which is the auxiliary space required for the lookup table

\section{ EXPERIMENTAL STUDY
}
\includegraphics[width=.48\textwidth]{3.png}

\begin{center}
	\begin{tabular}{ |c | c | }
	\hline
	
N(string length)&
Time( in seconds)



 \\
	 \hline

100&
0.022989

\\
	\hline

500&
0.031340

\\
\hline

1000&
0.055731

\\
\hline

2000&
0.132250

\\
\hline
	\end{tabular}
\end{center}

From the plotted graph of  test cases (string length:x-axis) against time taken(y-axis)  we can see the time complexity of the graph to be O(n^2)(by asymptotic analysis)

\section{ CONCLUSION
}
In this paper we proposed an solution to calculate the number of palindromic substring of a given string using dynamic programming approach.The approach has a time complexity of O(n^2) which is found through  study and asymptotic analysis of the graph
\section{ REFERENCES    

}


      [1]  https://www.geeksforgeeks.org
      [2]  Introduction to Algorithms (CLRS)

\end{document}
